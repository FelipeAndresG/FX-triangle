\documentclass[a4paper, 12pt]{article}
\usepackage[utf8]{inputenc}
\usepackage[english]{babel}
\usepackage[font=small,labelfont=bf]{caption}
\usepackage{lmodern,textcomp,ifthen,graphicx,enumitem,amsmath,amsfonts,booktabs,csvsimple}

\numberwithin{equation}{subsection}
\renewcommand{\thefigure}{\arabic{section}.\arabic{figure}}
\renewcommand{\thetable}{\arabic{section}.\arabic{table}}

\usepackage[notes,
            titlepage,
            a4paper,
            pagenumber,
            sectionmark,
            twoside,
            fancysections]{polytechnique}
\usepackage[colorlinks=true,
            linkcolor=black,%bleu303,
            filecolor=red,
            urlcolor=bleu303,
            bookmarks=true,
            bookmarksopen=true]{hyperref}
			
%% Tamanho figuras
\newlength{\mylength}
\setlength{\mylength}{0.5\textwidth}
\setlength{\parindent}{0pt}

%%%%%%%%%%%%%%%%%%
%% FIN COMANDOS %%
%%%%%%%%%%%%%%%%%%

\title{Calibration d’un triangle de taux de change}
\subtitle{Projet Recherche 3A}
\author{
\begin{tabular}{ll}
\\
Auteurs : & Lucas Furquim \\
		& Felipe Garc\'ia \\
Tuteur : & Benjamin Jourdain
\end{tabular}
\\}
\date\today

\begin{document}
    \maketitle
    \renewcommand{\baselinestretch}{1.1}
    \setlength{\parskip}{0.5em}
	

\pagebreak

\tableofcontents

\clearpage

\section{Project Description}

In this project we describe a calibration method based on \cite{guyon2013new} for three exchange rates with local volatility. for example between USD, EUR and GBP. We already know how to calibrate the two exchange rates $S^1 = $ USD/EUR and $S^2 = $ EUR/GBP. To calibrate the third exchange rate we need to take into account that $S^{12} = S^1 / S^2$. In this project we present a  model on how to calibrate the third exchange rate.

In this context, this project will provide a general theoretical analysis volatility regimes, considering implied volatility, local volatility models and models on exchange rates. We last show an equation \textbf{(\ref{calibration})} that shows us how to calibrate this three exchange rate model.

\section{General Background}

\subsection{Implied Volatility}
In the Black-Scholes model, options' price is uniquely determined by the only
unobservable parameter: the volatility. Knowing Put and Call market prices,
we can invert Black-Scholes equation and extract this parameter, named in this
case ``implied volatility"(IV).

\begin{align}
C = SN(d_{+}) - e^{-r(T-t)}KN(d_{-}),\: t \in [0, T] \\
d_{\pm} = \dfrac{\ln(\frac{S}{K} + (r \pm \sigma^2 / 2)(T-t))}{\sigma \sqrt[]{T-t}}\label{BS}
\end{align} 
In this framework, IV should be constant across all
strikes and maturities and equal to the historical volatility (standard deviation
of annualized log returns) of the underlying. However, when IV is computed
from market quoted option prices, one observes that:
 
\begin{itemize}
\item IV and historical volatility are generally different. When IV is greater than historical volatility, options are thought to be overvalued, and when IV is
less than historical volatility, options are considered to be undervalued;
\item IV of different options on the same underlying depend on their maturities
and strikes.
\end{itemize}

The implied volatility, then, becomes a non-constant function of strike, and the shape of the curve denotes a phenomenon known as volatility smile or skew (in equity speak).

Volatility skew can be used to identify trading opportunities. In practice, implied volatility allows traders to compare options with different strikes and expiration dates. The implied volatility enables traders to gain a better perspective of derivatives' market. Since supply and demand ultimately drive prices, traders
can learn which options are ``cheap" or ``expensive", relative to others, as measured by the implied volatility of each option.
\\

The parameter $\sigma$ in \textbf{(\ref{BS})} corresponds to the average volatility of the underlying asset during the lifetime of the contract. This is the only parameter in the Black  \& Scholes model that is not directly observable in the financial markets. In order to find the market price of a publicly traded option by using the Black \& Scholes model, it is necessary to find a matching value for $\sigma$. Generally, the value of $\sigma$ which produces a market price is called \textit{implied volatility}.\\


\textbf{Defintion. }\textit{The implied volatility is the volatility which makes \textbf{(\ref{BS})} generate a price consistent with the price of a market quoted call option.}\\

The implied volatility is found by inverting \textbf{(\ref{BS})} with respect to the volatility parameter, using an option price quoted on the financial markets with known strike and maturity. Since \textbf{(\ref{BS})} is not analytically invertible with respect to $\sigma$,the implied volatility has to be found using numerical techniques.\\

The implied volatility for a grid of market quoted options with different maturities and strikes is generally not constant. The strike and maturity dependency in implied volatility is caused by the financial markets attaching higher probabilities to extreme movements in log-returns compared to the normal distribution. As a function of strike, the implied volatility for FX-options usually assumes the \textbf{shape of a smile}. This curve is generally denoted the implied volatility smile.\\


It is possible to add maturity dependency in $\sigma$ while keeping a closed-form solution similar to \textbf{(\ref{BS})}. Assume that the volatility is a deterministic function of time, $\sigma(t)$. Assuming $\sigma(t)$ to be a step function allows for arbitrage-free calibration to market prices on options with any set of maturities. This simple approach is sadly not applicable in the strike dimension, which leads us to the next section.\\

\subsection{Local Volatility and Dupire Formula}
The local volatility model is a generalization of the Black \& Scholes model. The model was first proposed by Dupire (1994) and has been further developed by Derman and Kani (1998) among others. The model is based on the assumption that the volatility is a general deterministic function dependent on time and the contemporaneous value of the underlying asset. This generalization makes it possible to create a risk-neutral probability distribution of the underlying asset which is consistent with an entire market quoted implied volatility surface. \\

To emphasize one of the strengths of the local volatility model we need the following definition.\\

\textbf{Definition. } \textit{A model is complete if all contingent claims can be perfectly
hedged.}\\

Since the local volatility model does not introduce any further sources of risk,
the model is a complete market model where assets theoretically can be perfectly
hedged by using a continuously updated $\delta$-hedging strategy.\\

Assume that the underlying asset follows a stochastic process with the dynamics

\begin{align}\label{EQ1}
dS_t = S_t\sigma(S_t, t)dW^{\mathbb{Q}}_{t}, t \in [0,T]
\end{align} 

where $(W^{\mathbb{Q}}_{t} )_{t\in[0,T]}$ is a standard Brownian motion under the risk-neutral probability measure. The central equation in the local volatility model, the Dupire equation, is presented below together with a proof similar to the one by Dupire (1994).\\

\textbf{Theorem. }\textit{(Dupire Equation for European Call Options) Let $C(K, T)$
denote European call options with strikes $K \in \kappa$  , maturities $T \in \theta$ , and under-lying asset $S$. Assume that the price of $S$ follows the dynamics in \textbf{(\ref{EQ1})}, with initial condition $S_t = S$. The prices of the call options at time t will then satisfy the equation:}


\begin{align}\label{EQDupire}
	\frac{\partial C}{\partial T}(K,T) = \frac{1}{2}\sigma^2(K,T)K^2\frac{\partial ^ 2 C}{\partial K^2}(K,T)
\end{align} 
\textit{with boundary condition:}
\begin{align}
	 \lim_{T \to t}C(K,T) = (S - K)^+.
\end{align} 

While this equation looks very similar to the Black \& Scholes equation, the two equations have fundamentally different meaning. The Black \& Scholes equation describes the evolution of the price of any European contingent claim over time, holding claim-specific parameters such as strike $K$ and maturity $T$ constant. The Dupire equation describes the dynamics in a grid of European call option prices, holding the contemporaneous value of the underlying asset $S$ and time $t$ constant. The two equations are often referred to as the forward and backward equations. This convention originates from the fact that the Black \& Scholes equation has boundary conditions at the terminal date $t = T$, and hence needs to be solved backwards in time, while the Dupire equation has boundary conditions at $T = t$ and needs to be solved forward in maturity. The remaining part of this section consists of a proof of the Dupire equation.\\

Consider a European call option $C(S, t)$ with some underlying asset $S$, following
the dynamics in \textbf{(\ref{EQ1})}. Assume further that the call option has maturity $T$ and
strike $K$. Denote the risk-neutral density of the underlying at time $t$ as $\varphi(S, t)$.
The price of the option can then be calculated as the expected payoff under the risk-neutral probability measure:
\begin{align}
	 C(S,T) = E^{Q}{[(S_t - K)\textbf{1}_{S_t - K>0}|S_t = S]}
\end{align} 
Using the definition of expected value, this expression can be formulated as the integral
\begin{align}\label{CallDupire}
	 C(S,T) = \int _0^\infty (x - K) \textbf{1}_{x>K}\varphi(x,T)dx = \int _K^\infty (x - K)\varphi(x,T)dx,
\end{align} 

where $\varphi(x,T)$ is the probability density function of the underlying asset at maturity, conditioned on $S_t = S$. In order to continue the derivation, the following theorem, usually called the Fokker-Planck theorem, is needed.
\\

\textbf{Theorem. }(Fokker-Planck) \textit{Let \textbf{$X_t$} be a N-dimensional stochastic process with uncertainty driven by a M-dimensional standard Brownian motion \textbf{$W_t$}:}
\begin{align}
	 d X_t = \mu (X_t,t)dt + \sigma(X_t,t)dW_t,
\end{align}
where $\mu(X_t,t) = (\mu(X_t,t) \cdots \mu _n(X_t,t))$ is a N-dimensional drift vector and $\sigma (X_t,t)$ is a diffusion tensor. Then the joint probability function $f(x, t)$ satisfies the Fokker-Planck equation
\begin{align}
\frac{\partial f(x,t)}{\partial t} = -\sum_{i=1}^{N}\frac{\partial}{\partial x_i}[\mu_i(x,t)f(x,t)]+\sum_{i=1}^N\sum_{j=1}^N\frac{\partial}{\partial x_i \partial x_j}[D_{i,j}(x,t)f(x,t)],
\end{align}
where $D_{i,j} = \frac{1}{2}\sum_{k=1}^M\sigma_{i,k}(x,1)\sigma_{j,k}(x,1), 1\leq i,j \leq N.\\ $
Applying this theorem to the driftless one-factor framework which is of interest yields the Fokker-Planck equation
\begin{align}\label{EQimpFK}
	 \frac{\partial}{\partial t}(\varphi(S,t)) = \frac{1}{2}\frac{\partial ^2}{\partial S^2}(S^2\sigma^2(S,t)\varphi(S,t)).
\end{align}
\\

The solution of this equation will (in accordance with The Second Fundamental Theorem of Asset Pricing) be assumed to be unique provided that the probability distribution is restricted to the risk neutral one. In order to proceed, some differentials of the option price with respect to the strike $K$ and maturity $T$ are needed. We will from now on let $K$ and $T$ be variables and $S$ and $t$ static parameters. The notation for a call option is also changed to $C(K, T)$, but keep in mind that the price is still conditioned on $S_t = S$. Using the Leibniz integral rule on \textbf{(\ref{CallDupire})} and assuming that $\lim_{S \to \infty} \varphi(S,T) = 0$ yields:
\begin{align}
	 \frac{\partial C}{\partial K} (K,T) &= - \int_K^\infty \varphi(x,T)dx, \label{EQ11}\\
     \frac{\partial ^2 C}{\partial K^2}(K,T) &= \varphi(K,T),\label{EQ12}\\
     \frac{\partial C}{\partial T}(K,T) &= \int_K^\infty (x - K) \frac{\partial}{\partial T}(\varphi(x,T))dx. \label{EQ13}
\end{align}
Substituting \textbf{(\ref{EQimpFK})} at time $t = T$ into \textbf{(\ref{EQ13})} yields:
\begin{align}\label{EQinterm}
	 \frac{\partial C}{\partial T}(S,T) &= \int_K^\infty (x - K)(\frac{1}{2}\frac{\partial ^2}{\partial x^2}(x^2\sigma^2(x,T)\varphi(x,T))dx.
\end{align}
Integrating this expression by parts two times and using the equations \textbf{(\ref{EQ11})} and \textbf{(\ref{EQ12})} yields:
\begin{align}\label{EQ1Final}
	 \int_K^\infty (x - K)(\frac{1}{2}\frac{\partial ^2}{\partial x^2}(x^2\sigma^2(x,T)\varphi(x,T))dx = \frac{1}{2}\sigma ^ 2 (K,T)K^2 \frac{\partial^2 C}{\partial K^2}.
\end{align}
Substituting \textbf{(\ref{EQ1Final})} into \textbf{(\ref{EQinterm})} yields the Dupire PDE for European call options. The boundary condition of the above equation simply states that an
option with time to maturity $\tau = T -t = 0$ will have the value equal to the call
option payoff function:
\begin{align}
	\lim_{T \to t}C(K,T) = (S-K)^+.
\end{align}

\pagebreak
\section{Model for exchange rate options}
Assume that there are riskless assets USD and GBP in dollars
and British pounds sterling, respectively, with riskless rates of return $r^d$ and $r^f$. Because of
uncertainty about future exchange rates, the asset GBP does not appear riskless
to the dollar investor, nor does the asset USD appear riskless to the pound sterling investor; the choice of numeraire (dollar or pound sterling) determines which asset is riskless. Let’s take the point of view of the pound-sterling investor. Let $Y_t$ be the rate of exchange at time $t$ (that is, $Y_t$ is the number of British pounds that one dollar will buy at time $t$). In the simplest model, $Y_t$ behaves like a geometric Brownian motion, that is, it follows a stochastic differential equation
of the form
\begin{align}
dY_t = \mu Y_t dt + \sigma Y_t dW_t \label{GK}
\end{align}
where $W_t$ is a Brownian motion. Let $A_t$ and $B_t$ denote the share prices of the assets USD
and GBP, reported in units of dollars and British pounds, respectively, and normalized so that the time-zero share prices are both 1. Then
\begin{align}
A_t = e^{r^d t}\\ 
B_t = e^{r^f t}
\end{align}
The share price of US dolar at time $t$ in pounds sterling is $A_t Y_t$. Solving the stochastic
differential equation \textbf{(\ref{GK})} gives the explicit formula

\begin{align}\label{exchange}
A_{t} Y_{t} = Y_{0} \exp \left \{ r^d t + \mu t - \sigma ^{2} t /2 + \sigma W_{t} \right \} .
\end{align}

Now let $\mathbb{Q}^f$ be the risk neutral measure of the english pound. Under $\mathbb{Q}^f$, the discounted price of the asset USD is $\exp \left \{ -r^f t \right \} A_tY_t$, and therefore by equation \textbf{(\ref{exchange})} equals 
\begin{align}
\exp \left \{ -r^f t \right \} A_tY_t = Y_0 \exp \left \{ (r^d - r^f) t + \mu t - \sigma^2 t /2 + \sigma W_t \right \}
\end{align}

The second exponential is by itself a martingale, and the first exponential is nonrandom. Thus, in order that the product of the two exponentials be a martingale it must be that $r^d - r^f + \mu = 0$. So $\mu = r^f - r^d$. Therefore the exchange rate for $Y_t$ is given by
\begin{align}
dY_t = (r^f - r^d) Y_t dt + \sigma Y_t dW_t. \label{fx}
\end{align}
\pagebreak
\section{The FX triangle calibration problem}
Let $S^1$, $S^2$ be two FX rates, and $S^1 / S^2$. For example $S^1 = $ EUR/USD, $S_2 = $ GBP/USD, and $S^{12} = S^1/S^2 = $ EUR/GBP (the cross rate). Assuming that the implied volatility $S^1$, $S^2$ and $S^{12}$ until some maturity $T$ are known from the market, and that those surfaces are jointly arbitrage-free. They correspond to three local volatility surfaces that we denote by $\sigma_1(t,S^1)$, $\sigma_2(t,S^2)$, and $\sigma_{12}(t,S^{12})$. Assume the following model for the dynamics of $S^1$ and $S^2$:
\begin{align}\label{EQPRINCIPAL}
	d S_t^1 &= (r_t^d - r_t^1)S_t^1 \:dt + \sigma_1(t,S_t^1)S_t^1\:dW_t^1 \nonumber\\
    d S_t^2 &= (r_t^d - r_t^2)S_t^2 \:dt + \sigma_2(t,S_t^2)S_t^2\:dW_t^2 \nonumber\\
    d \left\langle W^1,W^2 \right\rangle_t &= \rho(t,S_t^1,S_t^2)\: dt 
\end{align}
\\
We know from the Ito's quotient rule that:
\begin{align}
\dfrac{dS^{12}_{t}}{S^{12}_{t}} = \frac{d S_t^1}{S_t^1} - \frac{dS_t^2}{S_t^2} + \left( \frac{dS_t^2}{S_t^2} \right)^2 - \frac{d S_t^1}{S_t^1}\frac{dS_t^2}{S_t^2} 
\end{align}
\\
So, from (\textbf{\ref{EQPRINCIPAL}}):
\begin{align}
\dfrac{dS^{12}_{t}}{S^{12}_{t}}  = (r_t^2 - r_t^1)\:dt + \sigma_1 \:dW_t^1 - \sigma_2 \: dW_t^2 + \sigma_2^2 \:dt -\sigma_1\sigma_2 \: d \left\langle W^1,W^2 \right\rangle_t
\end{align}
Since $d \left\langle W^1,W^2 \right\rangle_t = \rho(t,S_t^1,S_t^2)\:dt$:
\begin{align}
\dfrac{dS^{12}_{t}}{S^{12}_{t}}  = (r_t^2 - r_t^1) \:dt + \sigma_1 \:dW_t^1 - \sigma_2 \:dW_t^2 + \sigma_2^2dt -\rho\sigma_1\sigma_2dt
\end{align}
\begin{align}
\dfrac{dS^{12}_{t}}{S^{12}_{t}}  = (r_t^2 - r_t^1)dt + \sigma_1 \: \left(dW_t^1 - \rho\sigma_2 \:dt \right) - \sigma_2(dW_t^2 -\sigma_2dt )
\end{align}
\begin{align}
\dfrac{dS^{12}_{t}}{S^{12}_{t}} = (r^{2}_t - r^{1}_t) \:dt + \sigma_1 \left(t, S^{1}_t \right)dW_t^{1,f} - \sigma_2(t, S_t^{2})dW_t^{2,f}
\end{align}
where:
\begin{align}
W_t^{1,f} &= W_t^{1} - \int_0^t \rho(s, S_s^{1}, S_s^{2}) \sigma_2(s, S_s^{2}) \: ds \\
W_t^{2,f} &= W_t^{2} - \int_0^t \sigma_2(s, S_s^{2}) \: ds
\end{align}
We can see by Girsanov Theorem (using the Novikov's condition) that $W_t^{1,f}$ and $W_t^{2,f}$ are Brownian motions under the risk neutral measure $\mathbb{Q}^{f}$ defined by 
\begin{align}
\dfrac{d\mathbb{Q}^{f}}{d\mathbb{Q}} = \dfrac{S_T^{2}}{S_0^{2}} \exp \left( \int_0^{T} (r_t^{2} - r_t^{d}) \: dt \right)
\end{align}
Now as $W_t^{1,f}$ and $W_t^{2,f}$ are Brownian motions under the risk neutral measure $V$ we conclude that also is 
\begin{align}
W_t^{f} = \int_0^{t} \dfrac{\sigma_1(s,S_s^1)dW_s^{1,f} - \sigma_2(s, S_s^{2})dW_s^{2,f}}{a_s} \:ds
\end{align}
where $a_t^2 = \sigma_1^2(t, S_t^1) + \sigma_2^2(t, S_t^2) - \rho(t, S_t^1, S_t^2)\sigma_1^2(t, S_t^1)\sigma_2^2(t, S_t^2)$. Being a normalized linear combination of two Brownian Motions. Thus we have 
\begin{align}
\dfrac{dS^{12}_{t}}{S^{12}_{t}} = (r_t^{2} - r_t^{1}) dt + a_t dW_t^f
\end{align}
But we recall that we also have via Garman-Kohlhagen for the third exchange rate that
\begin{align}
\dfrac{dS^{12}_{t}}{S^{12}_{t}} = (r_t^{2} - r_t^{1}) dt + \sigma_{12}(t, S_t^{12}) dW_t^f
\end{align}
to end the proof we use the Gyongy's theorem from the appendix. Thus we have:
\begin{align}
\mathbb{E}(a_t^2 | S_t^{12}) = \sigma_{12}^2
\end{align}
which is equivalent to the calibration requirement
\begin{align}
\mathbb{E}^{\mathbb{Q}^f}_{\rho} \left[ \sigma_1^2(t, S_t^1) + \sigma_2^2(t, S_t^2) - 2\rho(t, S_t^1, S_t^2)\sigma_1^2(t, S_t^1)\sigma_2^2(t, S_t^2)|S_t^{12} \right] = \sigma_{12}^2(t, S_t^{12}) \label{calibration}
\end{align}
\\

From now on, we will denote $\mathcal{C}$ the set of functions $\rho$ : $[0,T] \times \mathbb{R}_+^* \times \mathbb{R}_+^* \to [-1,1]$. And any $\rho \in \mathcal{C}$ satisfying \textbf{(\ref{calibration})} will be called ``\textbf{admissible correlation}".
\\

Let us now pick two functions $a(t,S^1,S^2)$ and $b(t,S^1,S^2)$ such that b does not vanish and
\begin{align}\label{ABequation}
a(t,S^1,S^2) + b(t,S^1,S^2)\rho(t,S^1,S^2) \equiv f\left(t,\frac{S^1}{S^2}\right)
\end{align}
is local in cross.
Then:
\begin{align}
\sigma_{12}^2(t,\frac{S_t^1}{S_t^2}) &= \mathbb{E}_\rho^{\mathbb{Q}^f}\left[\sigma_1^2(t,S_t) + \sigma_2^2(t,S_t^2) - 2\rho(t,S_t^1,S_t^2)\sigma_1(t,S_t^1)\sigma_2(t,S_t^2)|\frac{S_t^1}{S_t^2}\right]
\nonumber\\
&= \mathbb{E}_\rho^{\mathbb{Q}^f}\left[\sigma_1^2(t,S_t^1) + \sigma_2^2(t,S_t^2) + 2\frac{a(t,S_t^1,S_t^2)}{b(t,S_t^1,S_t^2)}\sigma_1 (t,S_t^1)\sigma_2(t,S_t^2)|\frac{S_t^1}{S_t^2}\right] 
\nonumber\\
&\quad - 2(a+b\rho)\left(t,\frac{S_t^1}{S_t^2}\right)\mathbb{E}_\rho^{\mathbb{Q}^f}\left[\frac{\sigma_1(t,S_t^1)\sigma_2(t,S_t^2)}{b(t,S_t^1,S_t^2)}|\frac{S_t^1}{S_t^2}\right]
\end{align}
As consequence $\rho = \rho_{(a,b)}$ satisfies $\rho_{(a,b)} \in \mathcal{C}$ and
\begin{align}\label{rhoAB}
\rho_{(a,b)}(t,S_t^1,S_t^2) &= \frac{1}{b(t,S_t^1,S_t^2)} \nonumber\\ & \left(
 \frac{\mathbb{E}_{\rho(a,b)}^{\mathbb{Q}^f}\left[\sigma_1^2(t,S_t^1) + \sigma_2^2(t,S_t^2) + 2\frac{a(t,S_t^1,S_t^2)}{b(t,S_t^1,S_t^2)}\sigma_1 (t,S_t^1)\sigma_2(t,S_t^2)|\frac{S_t^1}{S_t^2}-\sigma_{12}^2(t,\frac{S_t^1}{S_t^2})\right]}{\mathbb{E}_\rho^{\mathbb{Q}^f}[\frac{\sigma_1(t,S_t^1)\sigma_2(t,S_t^2)}{b(t,S_t^1,S_t^2)}|\frac{S_t^1}{S_t^2}]} - a(t,S_t^1,S_t^2) \right)
\end{align}
We have thus proved that any admissible correlation is of the above type. Conversely, \textbf{if a function $\rho_{(a,b)} \in \mathcal{C}$ satisfies (\ref{rhoAB}), then it is an admissible correlation}. We call \textbf{(\ref{rhoAB})} the “\textbf{local in cross $a+b\rho$ representation}” of admissible correlations.
Thus we can rewrite \textbf{(\ref{EQPRINCIPAL})} as:
\begin{align}
	d S_t^1 &= (r_t^d - r_t^1)S_t^1 dt + \sigma_1(t,S_t^1)S_t^1dW_t^1 \nonumber\\
    d S_t^2 &= (r_t^d - r_t^2)S_t^2 dt + \sigma_2(t,S_t^2)S_t^2dW_t^2 \nonumber\\
    d \left\langle W^1,W^2 \right\rangle_t &= \frac{dt}{b(t,S_t^1,S_t^2)} \\ 
    &\left(\frac{\mathbb{E}_{\rho(a,b)}^{\mathbb{Q}^f}\left[\sigma_1^2(t,S_t^1) + \sigma_2^2(t,S_t^2) + 2\frac{a(t,S_t^1,S_t^2)}{b(t,S_t^1,S_t^2)}\sigma_1 (t,S_t^1)\sigma_2(t,S_t^2)|\frac{S_t^1}{S_t^2}-\sigma_{12}^2(t,\frac{S_t^1}{S_t^2})\right]}{\mathbb{E}_\rho^{\mathbb{Q}^f}[\frac{\sigma_1(t,S_t^1)\sigma_2(t,S_t^2)}{b(t,S_t^1,S_t^2)}|\frac{S_t^1}{S_t^2}]}- a(t,S_t^1,S_t^2)\right)
\end{align}
In practice, one may try to build a solution $\rho_{(a,b)} \in \mathcal{C}$ using the \textbf{particle method}, that can be described as follows. Let {$t_k$} denote a time discretization of $[0,T]$. We simulate $N$ processes $(S_t^{1,i},S_t^{2,i})_{1\leq i \leq N}$ starting from $(S_0^1,S_0^2)$ at time 0 using $N$ independent Brownian motions under the domestic measure $\mathbb{Q}$ as follows:
\begin{enumerate}
\item Initialize k = 1 and set $\rho_{(a,b)}(t,S_t^1,S_t^2) = \frac{\sigma_1^2(0,S^1)+\sigma_1^2(0,S^1)-\sigma_{12}^2(0,\frac{S^1}{S^2})}{2\sigma_1^2(0,S^1)\sigma_2^2(0,S^2)}$ for all $t \in [t_0 = 0; t_i]$
\item Simulate $(S_t^{1,i},S_t^{2,i})_{1\leq i \leq N}$ from $t_{k-1}$ to $t_k$ using a discretization scheme - say a \textbf{log-Euler scheme}
\item For all $S^{12}$ in a grid $G_{t_k}$ of cross rate values, compute
\begin{align}
E_{t_k}^{num}(S^{12}) &= \frac{\sum_{i=1}^{N}S_{t_k}^{2,i}\left(\sigma_1^2(t_k,S_{t_k}^{1,i}) + \sigma_2^2(t,S_{t_k}^{2,i}) + 2\frac{a(t,S_{t_k}^{1,i},S_{t_k}^{2,i})}{b(t,S_{t_k}^{1,i},S_{t_k}^{2,i})}\sigma_1 (t,S_{t_k}^{1,i})\sigma_2(t,S_{t_k}^{2,i})\right)\delta_{t_k,N}\left(\frac{S_{t_k}^{1,i}}{S_{t_k}^{2,i}} - S^{12}\right)}{\sum_{i=1}^N S_{t_k}^{2,i}\delta_{t_k,N}\left(\frac{S_{t_k}^{1,i}}{S_{t_k}^{2,i}} - S^{12}\right) }
\nonumber\\
E_{t_k}^{den}(S^{12}) &= \frac{\sum_{i=1}^{N}S_{t_k}^{2,i}\frac{\sigma_1^2(t_k,S_{t_k}^{1,i}\sigma_2^2(t,S_{t_k}^{2,i})}{b(t,S_{t_k}^{1,i},S_{t_k}^{2,i})}\delta_{t_k,N}\left(\frac{S_{t_k}^{1,i}}{S_{t_k}^{2,i}} - S^{12}\right)}{\sum_{i=1}^{N}S_{t_k}^{2,i}\delta_{t_k,N}\left(\frac{S_{t_k}^{1,i}}{S_{t_k}^{2,i}} - S^{12}\right)}
\nonumber\\
f(t_k, S^{12}) &= \frac{E_{t_k}^{num}(S^{12}) - \sigma^{12}(t_k,S^{12})}{2E_{t_k}^{den}(S^{12})}
\nonumber
\end{align}
interpolate and extrapolate $f(t_k,.)$, for instance using cubic splines, and, for all $t \in [t_k, t_{k+1}]$, set
\begin{align}
\rho_{(a,b)}(t,S^1,S^2)=\frac{1}{b(t,S^1,S^2)}\left(f\left(t_k,\frac{S^1}{S^2}\right)-a(t,S^1,S^2)\right)
\end{align}
\item Set $k := k +1$. Iterate steps 2 and 3 up the maturity date $T$.
\end{enumerate}

\clearpage
\section{Conclusion}
With the general theoretical analysis on volatility regimes and models on exchange rates proposed on this project, we were able to derivate equation \textbf{(\ref{calibration})} that shows us how to calibrate this three exchange rate model. Besides, we showed how to  build a solution using the particle method.\\
For the further development of our project, we intend to deepen our studies, apply and analyze the numerical algorithm proposed.


    \begin{center}
        \color{bleu303}

        \rule{0.3\textwidth}{0.2mm}\vspace*{-3.5mm}

        \rule{0.5\textwidth}{0.6mm}\vspace*{-3.8mm}

        \rule{0.3\textwidth}{0.2mm}\vspace*{-1mm}

        \sffamily FIN
    \end{center}

\clearpage
\section{Bibliography}
{
\renewcommand{\section}[2]{}
\nocite{*}
\bibliographystyle{alpha}
% * <felipe.a.garcia.s@gmail.com> 2016-11-30T10:09:30.156Z:
%
% > alpha
%
% ^.
\bibliography{main}
}

\pagebreak

\section{APPENDIX}


For the following theorems we denote
\begin{align}
Z_t^{(a)} = exp\left(\int_0^t a\phi_s dW_s - \frac{a^2}{2}\int_0^t|\phi_s|^2ds\right),   0 \leq t \leq T.
\end{align}
\textbf{Novikov's criterion.}
Suppose that
\begin{align}
E[e^{\frac{1}{2}\int_0^T |\phi_s|^2 ds}] < \infty
\end{align}
Then $E[Z_t] = 1$ and the process {$Z_t, 0 \leq t \leq T$} is a martingale. 
\\
\\
\textbf{Girsanov's Theorem.}
Suppose that $E[Z_t]=1$. Then the process
\begin{align}
\tilde{W} := W_t - \int_0^t\phi_sds, t \leq T
\end{align}
is a Brownian motion under the probability $\mathbb{Q} = Z_t\mathbb{P}$ on $\mathcal{F}_T$
\\
\\
\textbf{Gyongy's Theorem.}
Suppose
\begin{align}
dX(t) = \mu_tdt + \sigma_tdW(t),
\end{align}
where $\mu_t$ and $\sigma_t$ are adapted random processes and $W(t)$ is a Brownian motion. Define (nonrandom) functions
\begin{align}
\hat{\mu}(t,x) &= \mathbb{E}[\mu_t|X(t) = x],\\
\hat{\sigma}(t,x) &= \left(\mathbb{E}[\sigma_t^2|X(t)=x]\right)^\frac{1}{2}
\end{align}
Then there exists a solution of the stochastic differential equation
\begin{align}
dY(t) = \hat{\mu}(t,Y(t))dt + \hat{\sigma}(t,Y(t))dW(t)
\end{align}
with initial condition $Y(0) = X(0)$ such that at each fixed time t, $Y(t) \stackrel{\mathcal{D}}{=} X(t)$.
\\
\\
\textbf{Garman-Kohlhagen model.}
We consider a model with a domestic and a foreign country with domestic and foreign interest rates $r^d$ and $r^f$ and let $S$ be the exchange rate of the foreign under the domestic country. Then the following holds:
\begin{align}
dS_t/S_t = (r^f - r^d)\:dt + \sigma \:dW_t
\end{align}
\\
\textbf{Proposition. }
Let us consider the following dynamics for an asset S, where the volatility $a_t$, the interest rate $r_t$, and the repo $q_t$, inclusive of the dividend yield, are all stochastic processes:
\begin{align}
\frac{dS_t}{S_t}=(r_t - q_t)dt + a_tdW_t
\end{align}
This model is exactly calibrated to the market smile of $S$ if and only if
\begin{align}
\frac{\mathbb{E}[D_{0t}a_t^2|S_t = K]}{\mathbb{E}[D_{0t}|S_t = K]} = \sigma_Dup(t,K)^2 - \frac{\mathbb{E}[D_{0t}(r_t-q_t-(r_t^0 - q_t^0))\mathbf{1}_{S_t>K}]}{\frac{1}{2}K\partial_K^2 \mathcal{C}(t,K)} + \frac{\mathbb{E}[D_{0t}(q_t - q_t^0)(S_t - K)^+]}{\frac{1}{2}K^2\partial_K^2 \mathcal{C}(t,K)}
\end{align}
for all $(t,K)$, where $D_{0t} = exp\left(-\int_0^t r_sds\right)$ is the discount factor, $r_t^0$ and $q_t^0$ are deterministic rates and repos, and
\begin{align}
\sigma_{Dup}(t,K)^2 = \frac{\partial_t\mathcal{C}(t,K)+(r_t^0 - q_t^0)K\partial_K\mathcal{C}(t,K) + q_t^0\mathcal{C}(t,K)}{\frac{1}{2}K^2\partial_K^2 \mathcal{C}(t,K)}
\end{align}
with $\mathcal{C}(t,K)$ market price of the call option on $S$ with strike $K$ and maturity $t$.
\\
\textbf{REMARK} The deterministic rate $r_t^0$ is typically taken to be equal to $-\partial_t\ln P_{0t}$, with $P_{0t}$ the price at time 0 of a zero-coupon bond maturing at time $t$. Then one can infer a deterministic repo rate $q_{0t}$ from the forward price $f_0^t$:
\begin{align}
q_t^0 = r_t^0 - \partial_t\ln\frac{f_0^t}{S_0}
\end{align}
\end{document}